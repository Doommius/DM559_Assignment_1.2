%%% Do not change from BEGIN to END
%%% BEGIN
\documentclass[a4paper,10pt]{article}

\usepackage[utf8]{inputenc}
\usepackage[T1]{fontenc}
\usepackage[danish,english]{babel}
\usepackage{graphicx}
\usepackage[a4paper,margin=2.7cm]{geometry}
\usepackage[fleqn]{amsmath}
\usepackage[fleqn]{mathtools}

\usepackage[math]{kurier}

\usepackage{amsmath}
\usepackage{amssymb}
\usepackage{amsfonts}
\usepackage{enumerate}

\usepackage{tikz}
\usetikzlibrary{arrows,decorations.pathmorphing,backgrounds,positioning,fit,matrix}


\setlength{\parindent}{0pt}
\setlength{\parskip}{2ex} 


\renewcommand{\vec}[1]{\ensuremath {\mathbf #1}}

%%% Settings for the Instructor
\newcommand{\courseid}{DM559/DM545}
\newcommand{\coursename}{Linear and Integer Programming}
\newcommand{\term}{Spring 2017}
\newcommand{\dept}{Department of Mathematics and Computer Science}
%%%%%%%%%%%%%%%%%%%%%%%%%%%%%%%%%%%%%%%%%%%%%%%%%%%%%%%%%%%%

\usepackage{fancyhdr}
\usepackage{lastpage}
\usepackage{listings}
\lstset{language=Python,
  basicstyle=\ttfamily\lst@ifdisplaystyle\footnotesize\fi,%\fontfamily{pzc}\selectfont,%
  stringstyle=\ttfamily,
  commentstyle=\ttfamily,
  showstringspaces=false,
  frame=lines, 
  breaklines=true, tabsize=2,
  extendedchars=true,inputencoding=utf8
}
%\lstavoidwhitepre


\pagestyle{fancy} 
\lhead{{\sc \courseid -- \term }} 
\chead{}
\cfoot{Page \thepage\ of \pageref{LastPage}}

\fancypagestyle{plain}{
\lhead{\dept\\
University of Southern Denmark, Odense}
\chead{}
\rhead{\today\\
}
\lfoot{}
\rfoot{}
%\renewcommand{\headrulewidth}{0pt}
}


\title{\begin{flushleft}
\vspace{-4ex}
\courseid~-- \coursename \\[0.2cm]
{\Large Answers to Obligatory Assignment 1.2, \term \\[3ex]
\hrule}
\end{flushleft}
}


\date{}

%%% END

%%% Settings for the Student
%%%%%%%%%%%%%%%%%%%%%%%%%%%%%%%%%%%%%%%%%%%%%%%%%%%%%%%%%%%%
%%%%%%%%%%%%%%%%%%%%%%%%%%%%%%%%%%%%%%%%%%%%%%%%%%%%%%%%%%%%
\author{Osman Toplica, Mark Jervelund}
% You will be assigned an ID when you log in 
% in the submission system
%%%%%%%%%%%%%%%%%%%%%%%%%%%%%%%%%%%%%%%%%%%%%%%%%%%%%%%%%%%%
%%%%%%%%%%%%%%%%%%%%%%%%%%%%%%%%%%%%%%%%%%%%%%%%%%%%%%%%%%%%

\begin{document}
\maketitle

%%%%%%%%%%%%%%%%%%%%%%%%%%%%%%%%%%%%%%%%%%%%%%%%%%%%%%%%%%%%
%%%%%%%%%%%%%%%%%%%%%%%%%%%%%%%%%%%%%%%%%%%%%%%%%%%%%%%%%%%%
\section*{Task 1}
Our code for task 1\\
\lstinputlisting[language=python, frame=single,firstline=22, lastline=46,stepnumber=5,firstnumber=22]{../code.py}

and this is the output\\
\includegraphics[scale=0.4]{task1.png}

The runtime grows exponentially with the input size and 20 cities takes a few minutes. The runtime is slow because the algorithm creates all possible subtours $(2^n)$, then finds all possible paths and returns the path that is the shortest.
\newpage


\section*{Task 2}

The solution\\
\includegraphics[scale=0.4]{task_2.png}
\\

As seen in the plotted graph, we have 3 sub-tours, which could be expected since we have no constraint that makes this illegal. So the solution is \emph{not} a single tour. The matrix of the A graph is TU since the A graph is bipartite. All variables are integers, since we have no red lines, which is lucky. Testing with dantzig42 is another story:\\
\\
\includegraphics[scale=0.7]{task_2_2.png}

So non-integer variables are certainly possible in our implementation, but do not appear always.

\newpage



\section*{Task 3}
Here we've erased constraint number 2 from task 1 and ran the solver on an empty array. Since we've had sub-tours in the result, we've ran the solver with those sub-tours as input (overwriting the last value of the \emph{subtours} variable), and then once more.
\begin{lstlisting}
    #task 3
    subtours = []
    
    subtours = [[1,2,6,19, 14, 3, 10, 8, 11, 4, 7, 18],[0,16,17],[5,13,9,12,15]]
    
    subtours = [[1,2,6,19, 14, 3, 10, 8, 11, 4, 7, 18],[0,16,17],[5,13,9,12,15],
                [1,16,17,0,7,18], [4,8,11], [2,3,14,19,6], [5,10,13,9,12, 15]]

    tsplp_1 = solve_tsp(points, subtours)
    print tsplp_1
    tsputil.plot_situation(points, tsplp_1)
\end{lstlisting}

The output of each run\\
\includegraphics[scale=0.4]{task3_1.png}
\includegraphics[scale=0.4]{task3_2.png}\\
\includegraphics[scale=0.4]{task3_3.png}\\
\\
The last solution is identical to the one from task 1, which is what we want.

\newpage
\section*{Task 4}
We've defined a new variable, y and put it in the model instead of $z_i z_j$.\\
\\
$Max \sum\limits_{e=ij \in E' : i<j} x^*_e \cdot y \quad - \quad \sum\limits_{i \in V'\setminus \{k\}} z_i : z \in \mathbb{B}^n, z_k = 1$

$s.t.\quad y_{ij} \geq z_i+z_j - 1 \quad \text{for all} \quad ij \in E'$\\
$\phantom{s.t.} \quad y_{ij} \leq z_i\phantom{..............} \quad \text{for all} \quad ij \in E'$\\
$\phantom{s.t.} \quad y_{ij} \leq z_j\phantom{..............} \quad \text{for all} \quad ij \in E'$\\
$\phantom{s.t.} \quad 0 \leq y_{ij} \leq 1\phantom{........} \quad \text{for all} \quad ij \in E'$\\

The linear constraints enforce the relationship between y and $z_i z_j$ (i.e. y = 1 only if $z_i \land z_j $)


\section*{Task 5}
Our implementation of solve\_separation
\lstinputlisting[language=python, frame=single,firstline=133, lastline=158,stepnumber=5,firstnumber=133]{../code.py}

Running the function with tsplp\_0 and k = 1, we get the following output
\begin{lstlisting}
Separation problem solved for k=1, solution value 1
(1.0, [1, 2, 3, 4, 6, 7, 8, 10, 11, 14, 18, 19])
\end{lstlisting}
Which is as expected, if we compare it with the first result of task 3.

\newpage
\section*{Task 6}
The condition we've added 

\begin{lstlisting}
######### BEGIN: write here the condition. Include a tollerance
if best_val == 1:
######### END
\end{lstlisting}

We know that the solve\_separation function returns 1 if it has found a sub-tour of which k is part of. That is the only situation  in which we want to set found to true and continue the loop and find a better solution. When solve\_separation returns  0 for all k values, we know that there are no sub-tours and that an optimal solution has been found, so we can stop looping. We break out of the loop because the condition prevents the variable \emph{found} to be set to True when the optimal solution has been found.

The plot of the last solution found by solving $TSPLP_n$ \\
\includegraphics[scale=0.4]{task_6.png}

We know for a fact, that the solution from task 1 was optimal. Comparing this solution with the one from task 1, we can conclude that the solution is optimal, since they are identical.

\section*{Task 9}
The length of the optimal tour is 3284.9 as stated in previous tasks.


\newpage
\section*{Full source code}
Note: everything but the last tasks are outcommented.
\begin{lstlisting}
from gurobipy import *
from collections import OrderedDict
import tsputil
import math

from itertools import chain, combinations


def solve_tsp(points, subtours):
    points = list(points)

    V = range(len(points))
    E = [(i, j) for i in V for j in V if i < j]
    E = tuplelist(E)

    m = Model("TSP0")
    m.setParam(GRB.param.Presolve, 0)
    m.setParam(GRB.param.Method, 0)
    m.setParam(GRB.param.MIPGap, 1e-7)

    '''
    ######### BEGIN: Write here your model for Task 1
    x = {}
    for i,j in E:
        x[i, j] = m.addVar(vtype=GRB.BINARY)

    m.update()

    # 1.
    for i in V:
        m.addConstr(quicksum(x[i, j] for j in V if i < j) +
                    quicksum(x[j, i] for j in V if j < i) == 2)

    m.update()

    # 2.
    for s in subtours:
        m.addConstr(quicksum(x[i, j] for i, j in E if i in s if j in s) <= len(s)-1 )

    obj = quicksum(tsputil.distance(points[i], points[j]) * x[i,j] for i in V for j in V if i < j)

    m.setObjective(obj, GRB.MINIMIZE)

    m.update()

    ######### END

    ######### BEGIN: Write here your model for Task 2
    x = {}
    for i, j in E:
        x[i, j] = m.addVar(vtype=GRB.CONTINUOUS)
        x[i, j].lb = 0
        x[i, j].ub = 1

    m.update()

    # 1.
    for i in V:
        m.addConstr(quicksum(x[i, j] for j in V if i < j) +
                    quicksum(x[j, i] for j in V if j < i) == 2)

    m.update()

    obj = quicksum(tsputil.distance(points[i], points[j]) * x[i, j] for i in V for j in V if i < j)

    m.setObjective(obj, GRB.MINIMIZE)

    m.update()

    ######### END
    '''

    ######### BEGIN: Write here your model for Task 3
    x = {}
    for i,j in E:
        x[i, j] = m.addVar(vtype=GRB.BINARY)

    m.update()

    # 1.
    for i in V:
        m.addConstr(quicksum(x[i, j] for j in V if i < j) +
                    quicksum(x[j, i] for j in V if j < i) == 2)

    m.update()

    # 2.
    for s in subtours:
        m.addConstr(quicksum(x[i, j] for i, j in E if i in s if j in s) <= len(s)-1 )

    obj = quicksum(tsputil.distance(points[i], points[j]) * x[i,j] for i in V for j in V if i < j)

    m.setObjective(obj, GRB.MINIMIZE)

    m.update()

    ######### END

    m.optimize()
    m.write("tsplp.lp")

    if m.status == GRB.status.OPTIMAL:
        print('The optimal objective is %g' % m.objVal)
        m.write("tsplp.sol")  # write the solution
        return {(i, j): x[i, j].x for i, j in x}
    else:
        print
        "Something wrong in solve_tsplp"
        exit(0)




    return 0



# task 5
def solve_separation(points, x_star, k):
    points = list(points)
    V = range(len(points))
    Vprime = range(1, len(points))
    E = [(i, j) for i in V for j in V if i < j]
    Eprime = [(i, j) for i in Vprime for j in Vprime if i < j]
    E = tuplelist(E)
    Eprime = tuplelist(Eprime)

    m = Model("SEP")
    m.setParam(GRB.param.OutputFlag, 0)

    ######### BEGIN: Write here your model for Task 5
    y = {}
    for i, j in Eprime:
        y[i, j] = m.addVar(vtype=GRB.CONTINUOUS, lb=0, ub=1, obj=x_star[i, j])

    z = {}
    for i in Vprime:
        if i == k:
            z[i] = m.addVar(vtype=GRB.CONTINUOUS, lb=0, ub=1, obj=0)

        else:
            z[i] = m.addVar(vtype=GRB.CONTINUOUS, lb=0, ub=1, obj=-1)

    m.update()
    for i, j in Eprime:
        m.addConstr(y[i,j] >= z[i] + z[j] - 1)
        m.addConstr(y[i,j] <= z[i])
        m.addConstr(y[i,j] <= z[j])
    m.addConstr(z[k] == 1)

    obj = quicksum(x_star[i,j] * y[i,j] for i,j in Eprime if i < j) - quicksum(z[i] for i in Vprime if i != k)


    m.setObjective(obj, GRB.MAXIMIZE)

    ######### END
    m.optimize()
    m.write("sep.lp")

    if m.status == GRB.status.OPTIMAL:
        print('Separation problem solved for k=%d, solution value %g' % (k, m.objVal))
        m.write("sep.sol")  # write the solution
        subtour = filter(lambda i: z[i].x >= 0.99, z)
        return m.objVal, subtour
    else:
        print
        "Something wrong in solve_tsplp"
        exit(0)


        # return 0



#task 6
def cutting_plane_alg(points):
    Vprime = range(1,len(points))
    subtours = []
    found = True
    while found:
        lpsol = solve_tsp(points,subtours)
        tsputil.plot_situation(points, lpsol)
        found = False
        tmp_subtours = []
        best_val = float('-inf')
        for k in Vprime:
            value, subtour = solve_separation(points,lpsol,k)
            best_val = value if value > best_val else best_val
            ######### BEGIN: write here the condition. Include a tollerance
            if best_val == 1:
            ######### END
                found = True
                tmp_subtours += [subtour]
        subtours += tmp_subtours
        print ('*'*60)
        print ("********** Subtours found: ",tmp_subtours," with best value : ",best_val)
        print ('*'*60)
    tsputil.plot_situation(points, lpsol)

def powerset(iterable):
    "powerset([1,2,3]) --> () (1,) (2,) (3,) (1,2) (1,3) (2,3) (1,2,3)"
    s = list(iterable)
    return chain.from_iterable(combinations(s, r) for r in range(len(s) + 1))


def main(argv):
    n = 20
    seed = 1200
    points = tsputil.Cities(n, seed)

    #task 1
    #subtours = list(powerset(range(len(points))))
    # The first element of the list is the empty set and the last element is the full set, hence we remove them.
    #subtours = subtours[1:(len(subtours) - 1)]
    #tsplp = solve_tsp(points, subtours)
    #print tsplp
    #tsputil.plot_situation(points, tsplp)

    #task 2
    #tsplp_0 = solve_tsp(points, [])
    #print (tsplp_0)
    #tsputil.plot_situation(points, tsplp_0)
    #dantzig42 = tsputil.read_instance("dantzig42.dat")
    #tsplp_0 = solve_tsp(dantzig42, [])
    #print (tsplp_0)
    #tsputil.plot_situation(dantzig42, tsplp_0)

    #task 3
    #subtours = []
    #
    #subtours = [[1,2,6,19, 14, 3, 10, 8, 11, 4, 7, 18],[0,16,17],[5,13,9,12,15]]
    #
    #subtours = [[1,2,6,19, 14, 3, 10, 8, 11, 4, 7, 18],[0,16,17],[5,13,9,12,15],
    #            [1,16,17,0,7,18], [4,8,11], [2,3,14,19,6], [5,10,13,9,12, 15]]
    #
    #tsplp_1 = solve_tsp(points, subtours)
    #print tsplp_1
    #tsputil.plot_situation(points, tsplp_1)

    '''
    #task 4
    # modelling..
    '''

    #task 5
    #tsplp_0 = solve_tsp(points, [])
    #print solve_separation(points, tsplp_0, 1)
    #tsputil.plot_situation(points, tsplp_0)

    #task 6
    #dantzig42 = tsputil.read_instance("dantzig42.dat")
    cutting_plane_alg(points)

    #task 9
    # 3284.87

if __name__ == "__main__":
    main(sys.argv[1:])
\end{lstlisting}

\end{document}